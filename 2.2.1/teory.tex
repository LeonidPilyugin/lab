\section{Теоритические сведения}
Диффузия~--- самопроизвольное взаимное проникновение веществ друг в друга,
происходящее вследствие хаотичного теплового движения молекул. При перемешивании
молекул разного сорта говорят о взаимной диффузии. 

Диффузия в системе из двух компонентов подчиняется закону Фика:
\[j_a=-D\frac{\partial n_a}{\partial x}\]
$D$~--- коэффициент взаимной диффузии.

В работе исследуется взаимная диффузия гелия и воздуха. Давление и температура
неизменны:
\[p=\left(n_\text{He}+n_\text{в}\right)k_BT = const\]

Поэтому $\Delta n_\text{в} = - \Delta n_\text{He}$. Значит, достаточно описать диффузию одного
из компонентов. 

Концентрация и молярная масса гелия много меньше соответсвтующих параметров воздуха,
поэтому тепловая скорость молекул гелия велика по сравнению со скоростью молекул
воздуха. Значит, диффузию гелия можно описывать как диффузию примеси легких частиц 
гелия на фоне неподвижных молекул воздуха. В этом приближении коэффициент диффузии
равен
\[D=\frac{1}{3}\lambda \overline{v}\]
$\overline{v}=\sqrt{\frac{8RT}{\pi\mu}}$~--- средняя тепловая скорость частиц примеси.
$\lambda$~--- их длина свободного пробега. $\lambda = 1/n_0\sigma$,
$n_0$~--- концентрация рассеивающих центров, $\sigma$~--- сечение столкновения
частиц примеси с частицами фона.

Для бинарной смеси формула $D=\frac{1}{3}\lambda \overline{v}$ сохраняется, но
$\lambda=1/n_\Sigma\sigma$, $n_\Sigma=n_\text{He} + n_\text{в} = P/k_BT$, а
$\overline{v}$--- средняя отосительная скорость чатсиц.

Таким образом, коэффициент диффузии обратно пропорционален давлению и не зависит
от пропорций компонентов.

Для исследования взаимной диффузии газов используется два сосуда объемами $V_1\approx V_2=V$,
соединенные трубкой длины $L$ и сечением $S$. Сосуды заполенны заполнены смесью
двух газов при одинаковом давлении, но разной концентрации компонентов. Из-за диффузии
со временем концентрации компонентов выравниваются. 

Диффузия~--- медленный процесс и для его наблюдения необходимо отсутствие конвекции,
т.е. нужно обеспечить равенство давлений и температур в сосудах до начала измерений.

Объем трубки мал, поэтому концентрации в сосудах постоянны по всему объему и надо
исследовать их зависимости от времени $n_1(t)$ и $n_2(t)$.

Т.к. в трубке устанавливается стационарный поток, закон Фика для нее выглядит так:
\[j=-D\frac{\Delta n}{L}\]
$\Delta n$~--- разность концентраций гелия на концах трубки.

$N_1 = n_1V$ и $N_2=n_2V$~--- количества чатиц примеси в сосудах.
\[\frac{dN_1}{dt}=-\frac{dN_2}{dt}\]
\[\frac{d(\Delta n)}{dt}=-\frac{\Delta n}{\tau}\]
$\tau=VL/2DS$~--- характерное время диффузии.

\[\Delta n = \Delta n_0\exp\left(-\frac{t}{\tau}\right)\]

$\tau\gg L^2/2D$, поэтому процесс можно считать квазистационарным.

Влиянием сил тяжести можно пренебречь, т.к. $mgh\ll k_BT$.


