\section{Теоритические сведения}
Теплоёмкость тела связана с подведённым теплом $\Delta Q$ и изменением температуры $\Delta T$
соотношением \[C=\frac{\Delta Q}{\Delta T}\]

Идущая на нагрев тела теплота $\Delta Q$ равна
\[\Delta Q = P\Delta t - \lambda \left(T-T_{\text{к}}\right)\Delta t\]
$P$~--- мощность нагревателя, $\lambda$~--- коэффициент теплоотдачи,
$T$~--- температура тела, $T_{\text{к}}$~--- температура окружающей среды,
$\Delta t$~--- время нагревания.

Итого
\[C=\frac{P-\lambda\left(T-T_{\text{к}}\right)}{\Delta T/\Delta t}\]
$\Delta T = T - T_{\text{к}}$

С ростом температуры тела растёт утечка энергии и уменьшается скорость увеличения температуры.
