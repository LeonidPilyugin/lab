\section{Теоритические сведения}
Значительную часть радиационного фона составляет поток
космических частиц. Он изменяется во времени случайным образом.
Характеристиками этого потока являются среднее и СКО. Для их рассчёта
используются те же методы, что и для оценки случайных погрешностей.

Космические лучи разделяют на первичные, которые приходят из космоса, и
вторичные, которые возникают из-за взаимодействия первичных с землёй.
Вторичные лучи составляют основную часть фона у поверхности земли.

Первичные космические лучи~--- поток стабильных частиц с кинетической
энергией от $10^9$ до $10^{21}\,\text{эВ}$. В космическом пространстве
поток частиц изотропен. В основном космические лучи состоят из протонов.
Изменения интенсивности потока первичных частицневелики и  в основном
связаны с процессами в Солнце.

Первичные лучи попадают в атмосферу, взаимодействуют с её атомами и образуют
вторичные космические лучи.

Количество частиц в космических лучах характеризуется интенсивностью $I$~---
числом частиц, падающих в единицу времени на единичную площадку, перпендикулярную
к направлению наблюдения, отнесённое к единице телесного угла.

Плотность потока вторичных космических лучей вблизи поверхности Земли
сильно зависит от направления. Она максимальна в вертикальном и минимальна
в горизонтальном направлении. Её изменение приблизительно равно квадрату косинуса
угла отклонения от вертикали. Изменения плотности потока во времени вызваны
изменениями в атмосфере и магнитном поле.

Среднеквадратическая ошибка отсчётов, измеренная за некоторый интервал времени,
равна корню из измеренного значения $n$: $\sigma=\sqrt{n}$. Результат измерений
записывается как $n_0=n\pm\sqrt{n}$.

При $N$ измерениях среднее значение числа частиц:
\[\overline{n}=\frac{1}{N}\sum_{i=1}^N n_i\]
Стандартная ошибка измерения:
\[\sigma_\text{отд}=\sqrt{\frac{1}{N}\sum_{i=1}^{N}\left(n_i-\overline{n}\right)^2}\]
\[\sigma_\text{отд}\approx\sqrt{\overline{n}}\]
Стандартная ошибка отклонения $\overline{n}$ от $n_0$:
\[\sigma_{\overline{n}}=\frac{1}{N}\sqrt{\sum_{i=0}^N \left(n_i - \overline{n}\right)^2}=\frac{\sigma_\text{отд}}{\sqrt{N}}\]
Относительная ошибка отдельного измерения:
\[\varepsilon_\text{отд}=\frac{\sigma_{\overline{n}}}{n_i}\approx\frac{1}{\sqrt{n_i}}\]
Относительная ошибка в определении среднего по всем измерениям $\overline{n}$:
\[\varepsilon_{\overline{n}}=\frac{\sigma_{\overline{n}}}{\overline{n}}=\frac{\sigma_\text{отд}}{\overline{n}\sqrt{N}}\approx\frac{1}{\sqrt{\overline{n}N}}\]
