\section{Теоритические сведения}
Магнитометром называют прибор для магнитных измерений, например: компас, теодолит, веберметр и пр. С помощью магнитометров измеряют намагниченность ферромагнетиков, напряжённость магнитных полей, исследуют магнитные аномалии. Разработаны магнитометры различных конструкций: магнитостатические, электромагнитные, магнито-динамические, индукционные, резонансные. Эталонные магнитометры позволяют измерять горизонтальную и вертикальную составляющие напряжённости магнитного поля Земли с точностью $10^{-6}\;\text{Э}$.
