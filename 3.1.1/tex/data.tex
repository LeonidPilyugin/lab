\section{Результаты измерений}
\[
    m = \left(58800 \pm 5\right)\cdot 10^{-4}\;\text{г}
\]
\[
    L = 122{,}0 \pm 0{,}5\;\text{см}
\]
\[
    T = 11{,}4 \pm 0{,}5\;\text{с}
\]
\[
    l = 4{,}000 \pm 0{,}010\;\text{см}
\]
\[
   r = 2{,}50 \pm 0{,}10\;\text{мм}  
\]
\[
    R = 20{,}0 \pm 0{,}5\;\text{см}
\]
\[
    x_{1} = 14{,}5 \pm 1{,}0\;\text{см}
\]
\[
    x_{2} = \left(95 \pm 7\right)\cdot 10^{-4}\;\text{А}
\]
\[
    B_{0} = \left(71 \pm 5\right)\cdot 10^{-7}\;\text{Тл}
\]
\[
    C = 9{,}00\pm 0{,}18 \cdot 10^{5}\;\text{см} 
\]
\[
    U = 0{,}3\;\text{ед СГС}
\]
\[
    \nu = 50\;\text{Гц}
\]
\[
    I_{\left[\text{СГС}\right]} = \left(135 \pm 3\right)\cdot 10^{5}\;\text{ед СГС}
\]
\[
    I_{1} = \left(95 \pm 7\right)\cdot 10^{-4}\;\text{А}
\]
\[
    I_{2} = \left(110 \pm 8\right)\cdot 10^{-4}\;\text{А}
\]
\[
    c = 0{,}2 \frac{I_{\left[\text{СГС}\right]}}{I_{1} + I_{2}} = \left(132 \pm 9\right)\cdot 10^{8}\;\text{см} / \text{с} = 0{,}44 c_{0}
\]

Значение совпало по порядку, но все же отличается от $c$. Это может быть связано с изменением поля в момент измерения части 2 и тем, что луч, формирующий зайчик не перпендикулярен линейке.
