\section{Результаты измерений}

Толщина проволоки $d=1{,}06\pm 0{,}005\,\text{мм}$, длина $l=171\pm0{,}5\,\text{см}$.
Масса груза $m=204{,}4\pm0{,}1\,\text{г}$, длина $b=41\pm0{,}05\,\text{мм}$.

Показания счётчика записаны в таблице. $t_*$~--- время счётчика при $r=*\,\text{мм}$.
Показания счётчика связаны с номером колебания соотношением
\[t=n\tau\]
$\tau$~--- период колебания.

\newpage

\begin{table}[!ht]
    \centering
    \caption{Зависимость периода колебаний от $r$}
    \begin{tabular}{|l|l|l|l|l|l|}
    \hline
        $n$ & $t_{70},\,\pm 0{,}005\,\text{с}$& $t_{85},\,\pm 0{,}005\,\text{с}$ & $t_{100},\,\pm 0{,}005\,\text{с}$ & $t_{115},\,\pm 0{,}005\,\text{с}$ & $t_{130},\,\pm 0{,}005\,\text{с}$ \\ \hline
        $1$ & $2{,}78$ & $3{,}19$ & $3{,}44$ & $3{,}80$ & $4{,}20$ \\ \hline
        $2$ & $5{,}55$ & $6{,}30$ & $6{,}89$ & $7{,}53$ & $8{,}45$ \\ \hline
        $3$ & $8{,}20$ & $9{,}35$ & $10{,}31$ & $11{,}34$ & $12{,}59$ \\ \hline
        $4$ & $10{,}90$ & $12{,}42$ & $13{,}76$ & $15{,}07$ & $16{,}81$ \\ \hline
        $5$ & $13{,}69$ & $15{,}59$ & $17{,}20$ & $18{,}88$ & $21{,}04$ \\ \hline
        $6$ & $16{,}50$ & $18{,}76$ & $20{,}63$ & $22{,}60$ & $25{,}18$ \\ \hline
        $7$ & $19{,}31$ & $21{,}83$ & $24{,}08$ & $26{,}42$ & $29{,}42$ \\ \hline
        $8$ & $22{,}11$ & $24{,}88$ & $27{,}52$ & $30{,}14$ & $33{,}63$ \\ \hline
        $9$ & $24{,}91$ & $27{,}99$ & $30{,}95$ & $33{,}96$ & $37{,}78$ \\ \hline
        $10$ & $27{,}70$ & $31{,}18$ & $34{,}40$ & $37{,}96$ & $42{,}03$ \\ \hline
        $11$ & $30{,}48$ & $34{,}31$ & $37{,}84$ & $41{,}50$ & $46{,}23$ \\ \hline
        $12$ & $33{,}25$ & $37{,}36$ & $41{,}27$ & $45{,}24$ & $50{,}38$ \\ \hline
        $13$ & $36{,}02$ & $40{,}43$ & $44{,}72$ & $49{,}03$ & $54{,}64$ \\ \hline
        $14$ & $38{,}80$ & $43{,}58$ & $48{,}16$ & $52{,}78$ & $58{,}81$ \\ \hline
        $15$ & $41{,}58$ & $46{,}76$ & $51{,}59$ & $56{,}57$ & $62{,}99$ \\ \hline
        $16$ & $44{,}38$ & $49{,}84$ & $55{,}04$ & $60{,}32$ & $67{,}25$ \\ \hline
        $17$ & $47{,}19$ & $52{,}89$ & $58{,}48$ & $64{,}10$ & $71{,}40$ \\ \hline
        $18$ & $50{,}00$ & $55{,}99$ & $61{,}92$ & $67{,}87$ & $75{,}60$ \\ \hline
        $19$ & $52{,}80$ & $59{,}18$ & $65{,}37$ & $71{,}63$ & $79{,}85$ \\ \hline
        $20$ & $55{,}60$ & $62{,}31$ & $68{,}80$ & $75{,}41$ & $82{,}98$ \\ \hline
    \end{tabular}
\end{table}

\newcommand{\fuckingplot}[1]{\begin{figure}[h!]\center{\includegraphics{img/#1.png}}\end{figure}}

\fuckingplot{70}
\fuckingplot{85}
\fuckingplot{100}
\fuckingplot{115}
\fuckingplot{130}


\begin{gather*}
    \tau_{70} = 2{,}773\pm 0{,}006\,\text{с} \\
    \tau_{85} = 3{,}114\pm 0{,}007\,\text{с} \\
    \tau_{100} = 3{,}44\pm 0{,}005\,\text{с} \\
    \tau_{115} = 3{,}772\pm 0{,}005\,\text{с} \\
    \tau_{130} = 4{,}194\pm 0{,}006\,\text{с}
\end{gather*}

\[k=674\pm 2\;\text{с}^2/\text{м}^2\]
\[G=\frac{32lf}{\pi d^4}=\frac{256\pi ml}{kd^4}=\left(3{,}3\pm 0{,}06\right)\cdot 10^{11}\,\text{Па}\]
Я не нашёл такого значения $G$ в таблице.

\fuckingplot{main}
