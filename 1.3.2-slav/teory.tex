\section{Теоритические сведения}
При закручивании цилиндрических стержней круглого сечения
вдали от мест приложения моментов распределенгие деформаций
и напряжений одинаково. Для этих областей можно сичтать, что
каждое поперечное сечение поворачивается как жёсткое. Такое
напряжённое состояние называется чистым кручением. Касательные
напряжения в поперечном сечении увеличиваются пропорционально
расстоянию от оси вращения.

При рассмотрении закручиваемого цилиндра длины l можно заметить,
что любая прямая вертикальная линия, проведённая до закручивания
превращается в спираль. Сечения на расстоянии l повёрнуты на угол
$\varphi$.

Касательное напряжение $\tau$ связано с углом поворота соотношением
\[\tau = Gr\frac{d\varphi}{dl},\]
где $G$~--- модуль сдвига.

Суммарный момент сил, создаваемый касательными напряжениями
\[M=\pi G\frac{R^4}{2}\frac{d\varphi}{dl}\]
Этот момент не меняется по длине цилиндра, поэтому 
\[M=\pi G\frac{R^4}{2}\frac{\varphi}{l}=f\varphi,\]
где $f$~--- модуль кручения.
\[G=\frac{2lf}{\pi R^4}=\frac{32lf}{\pi d^4}\]
$d$~--- диаметр проволоки.
