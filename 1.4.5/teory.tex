\section{Теоритические сведения}
При рассмотрении колебаний струны можно пренебречь изгибными напряжениями.
У прямой натянутой струны сила натяжения значительно превышает силу тяжести.

Движение элементов струны связано с передачей ей импульса и изменением формы.
Натяжение струны стремится вернуть ее в прямолинейное положение.

Скорость распространения поперечной волны в струне
\[u=\sqrt{\frac{F}{\rho}}\]
$F$~--- сила натяжения, $\rho$~--- масса струны на единицу длины.

Длина волны при частоте $\nu$
\[\lambda=\frac{u}{\nu}\]

Частоты собственных колебаний
\[\nu_n=n\frac{u}{2l}\]
$l$~--- длина струны, $n$~--- число полуволн
