\section{Теоритические сведения}
Большинство тел стремится увеличить свой объём при изобарном повышении температуры.
Согласно принципу Ле-Шателье–Брауна, их адиабатическое расширение должно сопровождаться
понижением температуры. Однако для резины Эти эффекты имеют обратный знак.

Модуль Юнга рещины порядка $1\,\text{МПа}$, коэффициент Пуассона близок к $0{,}5$,
поэтому при растяжении вдоль оси объём почти не меняется. В отличии от обычных
твёрдых тел, резина растягивается за счёт переориентации и перемещения звеньев
полимерных цепочек, а не изменения расстояния между атомами.

Рассмотрим движение тонкой полосы длиной $l$ под действием внешней силы $f$.
Работа $\delta A$, совершаемая образцом
\[\delta A = -ddl + PdV\]
$P$~--- атмосферное давление. Последнее слагаемое мало по сравнению с первым.

Тогда первое начало термодинамики
\[dU=TdS+fdl\]
\[F=U-TS\]
\[dF=-SdT+fdl\]
$F$~--- свободная энергия.
\[f=\left(\frac{\partial F}{\partial l}\right)_T,\;\;S=-\left(\frac{\partial F}{\partial T}\right)_l\]
Соотношение Гиббса-Гельмгольца:
\[U(T, l) = F(T, l) - T\left(\frac{\partial F}{\partial T}\right)_l\]
Уравнение Максвелла:
\[\left(\frac{\partial F}{\partial T}\right)_l=-\left(\frac{\partial S}{\partial l}\right)_T\]
Для изотермического процесса
\[\delta Q = TdS = T\left(\frac{\partial S}{\partial l}\right)_Tdl=-T\left(\frac{\partial f}{\partial T}\right)_ldl\]
\[f=\left(\frac{\partial U}{\partial l}\right)_T - T\left(\frac{\partial S}{\partial l}\right)_T\]

Эта формула показывает, что упругие свойства вещества определяются не
только зависимостью его внутренней энергии от деформации, но и изменением его энтропии.
В большинстве твёрдых тел доминирующим является первое слагаемое, тогда как в резине
(а также в газах) преобладает второе.

Предположим, что внутренняя энергия резины зависит только от температуры.
Рассмотрим изотермическое растяжение резины. Тогда внутренняя энергия постоянна и
\[\delta Q = TdS=-fdl\]
\[Q = -\int fdl = -A_\text{внеш}\]
При растяжении резина выделяет тепло, а работа силы целиком идёт передаётся
окружающей среде в виде тепла.

Исследуем зависимость сил растяжения от температуры.
\[f=-T\left(\frac{\partial S}{\partial l}\right)_T\]
Как видно, упругие свойства резины связаны не с изменением её внутренней
энергии при растяжении, а с изменением её энтропии. Данная ситуация вполне
аналогична адиабатическому расширению идеального газа, только энтропия
резины при растяжении не возрастает, а убывает.

Далее из уравнения Максвелла:
\[f=T\left(\frac{\partial f}{\partial T}\right)_l\]
Это может быть выполнено, только если сила прямо пропорциональна температуре, то есть
уравнение состояния имеет вид
\[f(T, l) = \frac{T}{T_0}\overline{f}\left(\frac{l}{l_0}\right)\]
$\overline{f}\left(\frac{l}{l_0}\right)$~--- некоторая функция, зависящая только от
растяжения образца. Тоггда модуль Юнга должен быть прямо пропорционален температуре.

Теперь рассмотрим адиабатическое растяжение резины. Квазистатический (обратимый) процесс
является изоэнтропическим. Тогда из первого начала
\[dU=fdl\]
\[dU=C_ldT\]
$C_l$~--- теплоемкость при постоянном удлинении. Считая изменение температуры малым,
\[\Delta T = \frac{1}{C_l}\int_{l_0}^lfdl=\frac{A_\text{внеш}}{C_l}\]
Таким образом, при адиабатическом растяжении резина нагревается пропорционально
работе внешних сил.
\[\Delta S = C_l\ln\frac{T}{T_0}-\frac{1}{T_0}\int_{l_0}^l\overline{f}dl\]
При увеличении $l$ энтропия должна убывать, что компенсируется увеличением температуры.

\[\Delta S(\lambda) = -const \cdot \left(\lambda^2+\frac{2}{\lambda}\right)\]
$\lambda = l/l_0$
\[F(T,\lambda) = s_0E\cdot\frac{1}{3}\left(\lambda-\frac{1}{\lambda^2}\right)\]
$s_0$~--- площадь сечения недеформированного образца, $E=E_0\frac{T}{T_0}$~---
модуль Юнга.

Упругие свойства резины обусловлены её молекулярной структурой. Её молекулы представляют
собой длинный цепочки атомов, которые в нерастянутом состоянии свёрнуты в клубки. 
Переплетающиеся цепочки связаны химическими связями. На молекулярном уровне резина
представляет собой практически однородную упругую сетку.

При растяжении расстояния между атомами в цепочках и энергия взаимодействия почти
не меняются. Деформация происходит из-за переориентации и перемещения звеньев
цепочек, из-за чего клубки разворачиваются. При этом молекулы становятся
более упорядоченными, а энтропия снижается.

В адиабатическом процессе энтропия сохраняется, следовательно, ориентационное
уменьшение энтропии будет скомпенсировано её ростом за счёт
усиления хаотических колебаний атомов, то есть повышением температуры
образца. Если же растяжение изотермическое, то уменьшение энтропии произойдёт
за счёт выделения тепла в окружающую среду.

При малых растяжениях томы переплетающихся цепочек,
оказавшиеся соседними, взаимодействуют между собой слабыми электростатическими
(ван-дер-ваальсовыми) силами. Пока эти связи не разорваны, резина ведёт себя
как обычное твёрдое тело.

Даже при очень больших растяжениях не происходит разрыва молекулярных связей,
поэтому такие растяжения должны юыть обратимы. Если сразу после растяжения вернуть
резину квазистатически в исходное состояние, её температура должна принять исходное значение.
На практике неизбежно имеет место теплообмен с окружающей средой, поэтому
растяжение резины не является строго адиабатическим и обратимым. Если же
попытаться уменьшить теплообмен за счёт сокращения времени адиабатического растяжения,
это приводит к другим необратимым эффектам: становится
значимым внутреннее трение (вязкость) резины, возбуждаются затухающие
колебания среды и т.п. Таким образом, измеряемое приращение температуры
всегда будет несколько меньше теоретического, а при быстром возврате к исходному
растяжению температура резины окажется несколько выше исходной.
