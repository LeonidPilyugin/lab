\section{Результаты измерений}

\subsection{Поступательный маятник}

Масса маятника $M=2900\pm 5\,\text{г}$, длины нитей $L=220\pm 0{,}5\,\text{см}$,

\begin{table}[!ht]
    \caption{Скорость пули в первом опыте}
    \begin{tabular}{|l|l|l|}
    \hline
    $m\;\pm 0{,}001\,\text{г}$ & $x\;\pm 0{,}05\,\text{мм}$ & $v\;\text{м}/\text{с}$ \\ \hline
    $0{,}502$                  & $11{,}8$                   & $144\pm 1$             \\ \hline
    $0{,}509$                  & $12$                       & $144\pm 1$             \\ \hline
    $0{,}511$                  & $11{,}6$                   & $139\pm 1$             \\ \hline
    $0{,}51$                   & $11{,}4$                   & $137\pm 1$             \\ \hline
    \end{tabular}
\end{table}

\[v=141\pm 3\,\text{м}/\text{с}\]
Разброс результатов связан с различием скоростей, т.к. он превышает погрешность измерений.

\subsection{Крутильный маятник}

\begin{gather*}
    r=223\pm 0{,}5\,\text{мм}\\
    R=358\pm 0{,}5\,\text{мм}\\
    d=500\pm 0{,}5\,\text{мм}\\
    T_1=12{,}35\pm 0{,}1\,\text{с}\\
    T_2=9{,}24\pm 0{,}1\,\text{с}\\
    M_1=730{,}3\pm 0{,}05\,\text{г}\\
    M_2=730{,}5\pm 0{,}05\,\text{г}\\
    M=\frac{M_1+M_2}{2}=730{,}4\pm 0{,}05\,\text{г}\\
    \sqrt{kI}=0{,}216\pm 0{,}008\,\text{кг}\cdot\text{м}^2/\text{с}
\end{gather*}

\begin{table}[ht!]
    \caption{Скорость пули во втором опыте}
    \begin{tabular}{|l|l|l|}
    \hline
    $m\;\pm 0{,}001\,\text{г}$ & $x\;\pm 0{,}5\,\text{мм}$ & $v\;\text{м}/\text{с}$ \\ \hline
    $0{,}51$                   & $100$                     & $190\pm 8$             \\ \hline
    $0{,}503$                  & $104$                     & $200\pm 8$             \\ \hline
    $0{,}514$                  & $98$                      & $185\pm 8$             \\ \hline
    $0{,}508$                  & $100$                     & $191\pm 8$             \\ \hline
    \end{tabular}
\end{table}

\[v=192\pm 5\,\text{м}/\text{с}\]
Разброс результатов связан с погрешностью измерений, т.к. среднее значение лежит в пределах погрешности всех измерений.
