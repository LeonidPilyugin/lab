\section{Теоритические сведения}

Большие скорости тела (пули) удобно определять по передаваемому ими импульсу.
При столкновении пули с другим телом в момент их удара можно пренебречь внешними
силами и применять закон сохранения импульса.

Если пуля застревает в тяжёлом теле, то их дальнейшая скорость мала и её легко
измерить другими методами. Время соударения можно оценить по глубине проникновения.
Если оно мало, то можно использовать закон сохранения импульса.

Для измерения импульса пули можно использовать баллистический маятник (маятник, колебания
которого вызываются коротким начальным импульсом). При этом время соударения должно быть
много меньше периода. При этом отклонение $\Delta\varphi$ за время соударения $\tau$ тоже
мало и связано с периодом $T$ и амплитудой $\varphi_m$:
\[\frac{\Delta\varphi}{\varphi_m}\approx\frac{2\pi\tau}{T}\]

Связь между максимальным отклонением и начальной скоростью описывается законом сохранения
энергии. Импульс пули определяется по максимальному начальному отклонению.

Колебания должны происходить в одной плоскости, а воздушная струя, выходящая вслед за
пулей, не должгна действовать на маятник.

\newpage
