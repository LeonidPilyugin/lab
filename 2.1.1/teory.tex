\section{Теоритические сведения}
Теплоёмкость газов неудобно измерять в калориметрах, т.к. их теплоёмкость много меньше теплоёмкости
калориметра, из-за чего измерения неточны. Для увеличения количества нагреваемого газа
его пропускают через калориметр, внутри которого установлен нагреватель. При этом
измеряются мощность нагревателя, масса воздуха, протекающего в единицу времени и
приращение его температуры.

Тогда удельная теплоёмкость газа при постоянном давлении $с_p$ равна
\[c_p=\frac{N-N_\text{пот}}{q\Delta T}\]
$N$~--- мощность нагревателя, $N_\text{пот}$~--- мощность потерь, $q$~--- расход газа,
$\Delta T$~--- приращение его температуры.
