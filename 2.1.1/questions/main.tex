\documentclass[a4paper, 12pt]{article}

\usepackage[english,russian]{babel}
\usepackage[left=2cm, top=2cm, right=2cm, bottom=2.5cm]{geometry}
\usepackage{amssymb}
\usepackage{amsmath}
\usepackage{graphicx}
\usepackage[tableposition=top,singlelinecheck=false]{caption}

\title{Анализ данных, полученных при выполнении работы. Расчёт теплоёмкости воздуха и коэффициента теплоотдачи калориметра.}
\author{Леонид Пилюгин, Б02-212}

\begin{document}
    \maketitle
    \section{Общие замечания}
    \begin{enumerate}
        \item $q=\rho\frac{dV}{dT}$~--- объемный расход воздуха
        \item $\rho=\frac{\mu p}{RT}\approx 1{,}2\,\text{кг}/\text{м}^3$
        \item ЭДС термопары $\varepsilon=\beta\Delta T$, $\beta=40{,}7\,\text{мкВ}/\text{К}$
        \item $N_\text{пот}=\alpha\Delta T$
        \item $N=(c_pq+\alpha)\Delta T$ (При фиксированном расходе мощность и температура пропорциональны)
        \item Сопротивление нагревателя около 35 Ом
        \item $I_0=\sqrt{N_0/R_\text{Н}}$
    \end{enumerate}
    \section{Проведение измерений}
    \begin{enumerate}
        \item $I_{max}\approx2{,}5 I_0$
        \item $\Delta T \propto N \propto  I^2$
        \item Измерения для 4-5 точек $\Delta T$ от 2 до 10 градусов
        \item Измерения для 2 разных расходов воздуха
    \end{enumerate}
    \section{Обработка}
    \begin{enumerate}
        \item Графики $\Delta T(N)$ для каждого расхода воздуха $q$
        \item Проверьте, что выполняется предположение о том, что
        тепловые потери пропорциональны разности температур,
        аппрксимируя прямой $y=kx$ найти $k$ для каждого расхода
        \item Проанализируйте зависимость $k(q)$ и использую формулу
        $N=(c_pq+\alpha)\Delta T$, определите значение теплоемкости
        воздуха при постоянном давлении ($c_p=2{,}5R/\mu\approx715\,\text{Дж}/\text{кг}$ или $2{,}5R=20{,}8$),
        определите долю потерь $N_\text{пот}/N$ в опыте
        \item Коэффициент теплоотдачи $\alpha$
    \end{enumerate}
\end{document}