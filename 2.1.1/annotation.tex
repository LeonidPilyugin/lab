\section{Аннотация}
\textbf{Цель работы:} измерить повышение температуры воздуха в зависимости от мощности
подводимого тепла и расхода при стационарном течении через трубу; исключив теп-
ловые потери, по результатам измерений определить теплоёмкость воздуха при посто-
янном давлении.

\textbf{Оборудование:} теплоизолированная стеклянная трубка; электронагрева-
тель; источник питания постоянного тока; амперметр, вольтметр (цифровые мульти-
метры); термопара, подключенная к микровольтметру; компрессор; газовый счётчик;
секундомер. 