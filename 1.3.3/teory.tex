\section{Теоритические сведения}
Сила вязкого трения описывается законом Ньютона:
\[\tau_{xy}=-\eta\frac{\partial u_x}{\partial y}\]
$\eta$~--- коэффициент вязкости.

Объемный расход $Q$~--- объем газа, протекающий через сечение трубы в единицу времени.
$Q$ зависит от перепада давления $\Delta P$, плотности $\rho$ и вязкости $\eta$ газа,
радиуса $R$ и длины $L$ трубы.

Движение газа может быть ламинарным (поле скоростей образует набор непрерывных линий тока)
и турбулентным (образуются вихри, слои жидкости перемешиваются, появляются существенные
флуктуации скорости течения и давления).

Характер движения определяется числом Рейнольдса
\[\mathrm{Re}=\frac{\rho u a}{\eta}\]
$\rho$~--- плотность среды, $u$~--- характерная скорость потока,
$\eta$~--- коэффициент вязкости, $a$~--- размер, на котором существенно меняется скорость течения.

При малых Re доминируют силы вязкости и течение ламинарно, при больших~--- турбулентно.
Переход к турбулентному течению происходит при $\mathrm{Re}\approx 1000$. 

При малых перепадах давления и скорсотях много меньше скорости звука
газ можно считать несжимаемым. В работе это выполняется.

Расход газа при ламине=арном течении описывается формулой Пуазейля:
\[Q=\frac{\pi R^4 \Delta P}{8\eta l}\]
При этом распределение скоростей неоднородно. Пуазейлевский профиль
устанавливается на расстоянии
\[l_0\approx 0{,}2R\cdot\mathrm{Re}\]
Если длина трубки меньше $l_0$, то действием сил трения можно пренебречь
и движение газа описывается уравнением Бернулли.

Расход в турбулентном течении примерно описывается формулой
\[Q=R^{5/2}\sqrt{\frac{\Delta P}{\rho l}}\]
