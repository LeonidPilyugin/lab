\section{Теоритические сведения}
Уравнения движения твёрдого тела:
\begin{gather*}
    \frac{d\vec{P}}{dt} = \vec{F} \\
    \frac{d\vec{L}}{dt} = \vec{M}
\end{gather*}

\[
    \vec{L} = \vec{\imath}\,I_x\omega_x + 
              \vec{\jmath}\,I_y\omega_y +
              \vec{k}\,I_z\omega_z
\]

$I_x,\;I_y,\;I_z$~--- главные моменты иннерции, $\omega_x,\;\omega_y,\;\omega_z$~---
компоненты $\vec{\omega}$ Для быстро вращающегося тела одна из компонент $\vec{L}$
значительно превышает остальные.

Угловая скорость прецессии связана с моментом внешних сил и моментом импульса гироскопа:
\[\vec{M}=\left[\vec{\Omega},\vec{L}\right]\]

Скорость прецессии для гироскопа с моментом иннерции $I$, угловой скоростью $\omega$,
грузом массой $m$ на расстоянии $l$ от оси вращения
\[
    \Omega = \frac{mgl}{I\omega_0}
\]
