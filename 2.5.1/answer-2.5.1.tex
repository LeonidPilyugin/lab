\documentclass[a4paper, 12pt]{article}

\usepackage[english,russian]{babel}
\usepackage[left=2cm, top=2cm, right=2cm, bottom=2.5cm]{geometry}
\usepackage{indentfirst}
\usepackage{enumitem}
\usepackage[tableposition=top,singlelinecheck=false]{caption}
\usepackage{subcaption}
\usepackage{graphicx}
\usepackage{wrapfig}
\usepackage{amssymb}
\usepackage{amsmath}
\usepackage{tabularx}

\begin{document}
    \begin{enumerate}
        \item $\Delta p = \sigma\left(\frac{1}{a} +\frac{1}{b}\right)$, $a$ и $b$~--- длины полуосей сечения
        (пузырек будет иметь форму эллипсоида вращания сечения капилляра вокруг меньшего диаметра)
        \item \[2\pi r \sigma = \pi r^2 \rho g h\]
        \[h = \frac{2\sigma}{rg\rho}\approx 3\,\text{см}\]
        \item Опускаю иглу так, чтобы она касалась поверхности воды. Верхний конец иглы
        открыт в атмосферу. Воздух над поверхностью воды немного разрежается.
        Атмосфера начинает выдавливать пузырек. Сначала он становится полусферой с радиусом иглы.
        При дальнейшем расширении он отделяется из-за ограниченности пленки у иглы. Перепад
        давления равен $\Delta p = \frac{2\sigma}{r}$, откуда нахожу радиус иглы (измеряя перепад давлений и зная из таблицы $\sigma$ спирта).
        \item Погрешность измерения диаметра микроскопом ($\varepsilon\approx 0{,}03$)
        \item При критической температуре $\sigma=0$, $T_\text{кр}=T + \sigma / k = 700\pm 10\,\text{К}$,
        $T$~--- некоторая точка с графика, $\sigma$~--- поверхностное натяжение при этой температуре,
        $k$~--- наклон графика $\sigma(T)$
    \end{enumerate}
\end{document}
