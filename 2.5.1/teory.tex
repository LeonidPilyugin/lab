\section{Теоритические сведения}

Наличие поверхностного слоя приводит к различию давлений по разные стороны от искривленной границы раздела двух сред.  Для сферического пузырька с воздухом  внутри жидкости избыточное давление даётся формулой Лапласа
\[\Delta p = \frac{2\sigma}{r},\]
где $\sigma$ --- коэффициент поверхностного натяжения, $r$ --- радиус кривизны поверхности раздела двух фаз.
Эта формула лежит в основе предлагаемого метода определения коэффициента поверхностного натяжения жидкости. Измеряется давление
$\Delta p$, необходимое для выталкивания в жидкость пузырька воздуха.
