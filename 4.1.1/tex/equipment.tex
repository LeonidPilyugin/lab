\section{Оборудование}
Оптическая скамья с осветителем, набор линз, экран и зрительная труба позволяют определить параметры оптических систем всеми описанными способами. Все оптические элементы устанавливаются на скамье при помощи рейтеров. Предметом служит транспарант с изображением вертикальной стрелки $l = 2\;\text{см}$, закрепленный на стекле осветителя. Ирисовая диафрагма позволяет менять величину поля зрения. Яркость зрения регулируется ручкой трансформатора осветителя.

Важную роль играет правильная центровка элементов. Проходя через плохо отцентрированную систему, лучи света могут отклониться и пройти мимо экрана или глаза наблюдателя. Центрировать линзы следует как по высоте, так и в поперечном сечении.`
