\section{Теоритические сведения}
Физический маятник~--- твёрдое тело, которое под действием силы тяжести может
свободно качаться вокруг неподвижной горизонтальной оси. Движение маятника
описывается уравнением
\[I\frac{d^2\varphi}{dt^2}=M\]
$I$~--- момент иннерции, $\varphi$~--- угол отклонения, от положения равновесия,
$t$~--- время, $M$~--- момент сил, действующих на маятник.

Затухающие колебания характеризуются декрементом затухания:
\[\gamma = T\ln\frac{\varphi_0}{\varphi_1}\]
$T$~--- период колебаний, $\varphi_0$~--- амплиткда в начале колебания
$\varphi_1$~--- амплитуда в конце колебания.

$\tau=1/\gamma$~--- время, за которое амплитуда колебаний уменьшается в $e$ раз.
Добротностью называется величина
\[Q=\pi\frac{\tau}{T}\]
