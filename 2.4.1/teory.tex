\section{Теоритические сведения}
Испарение --- переход вещества из жидкого в газообразное состояние. При нем с поверхности
вылетают молекулы, образуя пар. Вылетая, они преодолевают силы молекулярного сцепления. Также
при испарении совершается работа против внешнего давления $P$, т.к. объем жидкости меньше объема пара.
Такой переход совершают только частицы с достаточной энергией, поэтому испарение приводит к охлаждению
жидкости. Для сохранения температуры, к жидкости надо подводить тепло. Количество теплоты,
необходимое для изотермического испарения одного моля жидкости при внешнем давлении, равном
упругости ее насыщенных паров, называется молярной теплотой испарения.

Теплоту испарения можно измерить непостредственно, но так нельзя получить точный результат
из-за неконтролируемых потерь тепла. В данной работе используется косвенный метод,
основанный на формуле Клапейрона-Клаузиуса:
\[\frac{dP}{dT}=\frac{L}{T\left(V_2-V_1\right)}\]
$P$ --- давление насыщенного пара при температуре $T$, $L$ --- теплота испарения,
$V_2$ --- объем пара, $V_1$ --- объем жидкости. Найдя из опыта $dP/dT$, $T$, $V_1$ и $V_2$,
можно получить $L$ путем расчета. Величины $L$, $V_1$ и $V_2$ должны относиться к одному
количеству вещества (1 молю).

В опыте измерения производятся при давлениях, меньше атмосферного. 

Параметры измеряемого вещества (воды):
\begin{center}
    $T_\text{кип}=373\;\text{K}$\\
    $V_1 = 18\cdot 10^{-6}\;\text{м}^3/\text{моль}$\\
    $V_2 = 31\cdot 10^{-3}\;\text{м}^3/\text{моль}$\\
    $b = 26\cdot 10^{-6}\;\text{м}^3/\text{моль}$\\
    $a = 0{,}4\;\text{Па}\cdot\text{м}^6/\text{моль}^2$\\
\end{center}
$a$ и $b$ --- параметры газа Ван-дер-Ваальса:
\[\left(P+\frac{a}{V^2}\right)\left(V-b\right)=RT\]

$V_1$ очень мало по сравнению с $V_2$, поэтому им можно пренебречь.
Газ можно описывать идеальным: $PV=RT$.
Тогда
\[L=\frac{RT^2}{P}\frac{dP}{dT}=-R\frac{d\ln P}{d 1/T}\]

Температура жидкости измеряется термометром, давление пара определяется манометром,
$d\ln P/d1/T$ находится из наклона графика в соответствующих координатах.
