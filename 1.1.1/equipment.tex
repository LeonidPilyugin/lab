\section{Оборудование и инструментальные погрешности}
\textbf{Линейка:} $\Delta_{\text{л}}=\pm 0{,}5\,\text{мм}$

\textbf{Штангенциркуль:} $\Delta_{\text{шт}}=\pm 0{,}05\,\text{мм}$

\textbf{Микрометр:} $\Delta_{\text{м}}=\pm 0{,}005\,\text{мм}$

\textbf{Вольтметр:}
\begin{table}[h!]
    \begin{tabular}{|l|l|}
    \hline
    Система                   & Цифровая              \\ \hline
    Класс точности            & $0{,}5$               \\ \hline
    Цена деления              & $0{,}1\,\text{мВ}$    \\ \hline
    Внутреннее сопростивление & $10\,\text{МОм}$      \\ \hline
    Погрешность               & $0{,}1\,\text{мВ}$      \\ \hline
    \end{tabular}
\end{table}
\newpage
\textbf{Амперметр:}
\begin{table}[h!]
    \begin{tabular}{|l|l|}
    \hline
    Система                   & Магнито-электрическая \\ \hline
    Класс точности            & $0{,}5$               \\ \hline
    Предел измерений          & $750\,\text{мА}$      \\ \hline
    Число делений             & $150$                 \\ \hline
    Цена деления              & $5\,\text{мА}$         \\ \hline
    Чувствительность          & $30$                  \\ \hline
    Внутреннее сопростивление & $37\,\text{мОм}$      \\ \hline
    Погрешность               & $1\,\text{мА}$          \\ \hline
    \end{tabular}
\end{table}

\textbf{Мост постоянного тока Р4833:}
\begin{table}[h!]
    \begin{tabular}{|l|l|}
    \hline
    Класс точности                                & $0{,}1$                                               \\ \hline
    Разрядность магазина сопротивлений            & $5\,\text{ед.}$                                       \\ \hline
    Исследуемый диапозон измерений                & $10^{-4}$--$10,\text{Ом}$ (для множителя $N=10^{-2}$) \\ \hline
    Погрешность измерений в исследуемом диапозоне & $\pm 10\,\text{мОм}$                                  \\ \hline
    \end{tabular}
\end{table}

В диапозоне $R_0$ от $1$ до $10\,\text{Ом}$ относительная поправка к сопротивлению (отлличие $R'$ от $R$)
состоавляет около $10^{-6}$ ($R_0 = 10\,\text{Ом}$), чем можно пренебречь и считать, что $R'\approx R$.
